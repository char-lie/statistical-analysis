\chapter*{Вступ}
\addcontentsline{toc}{chapter}{Вступ}

\textit{Об’єкт дослідження} --- часові ряди та методи їх прогнозування.

\textit{Предмет дослідження} --- якість прогнозування часових рядів
різними методами.

Завдання:
\begin{enumerate}
  \item
    зробити прогноз часового ряду на основі ретроданих;
  \item
    зробити прогноз часового ряду на основі інших рядів,
    від яких він залежить;
  \item
    порівняти ці два підходи.
\end{enumerate}
Обов'язкове використання методу експоненційного зглажування
для побудови тренду часових рядів
та методу авторегресії для оцінки невипадкових помилок.
